\documentclass{article}
\usepackage{graphicx}                            % Required for inserting images
\usepackage[paper=a4paper, top=1.5cm, bottom=1.5cm, left=2.0cm, right=2.0cm, heightrounded]{geometry}
\usepackage{hyperref}                            % Clickable links (disabled for minimal TeX install)
\usepackage{array}
\usepackage{multirow}
\usepackage{longtable}
\usepackage{xcolor}
\usepackage{tabularx}
\usepackage{colortbl}
\usepackage{adjustbox}
\usepackage{rotating}
\usepackage{float}
\usepackage[justification=centering]{caption}
\usepackage{listings}
\usepackage{listings}

\usepackage{framed}
\usepackage[nobreak=true]{mdframed}
\usepackage{tikz}
\usepackage{longtable}
\usepackage{pdflscape}
\usepackage{graphicx}
\usepackage{pdfpages}

\usetikzlibrary{positioning, arrows.meta, shapes, calc}

\setlength{\parindent}{0pt}
\setlength{\parskip}{5pt}


\lstset{
  basicstyle=\ttfamily\small,
  frame=single,
  columns=fullflexible,
  keepspaces=true,
  showstringspaces=false,
  commentstyle=\color{gray},
  keywordstyle=\color{blue},
  numbers=left,
  numberstyle=\tiny\color{gray},
  stepnumber=1,
  numbersep=8pt,
  backgroundcolor=\color{gray!10},
  tabsize=2,
  captionpos=b
}
\graphicspath{ {./img/} }



\begin{document}


\begin{titlepage}
    \begin{center}
        {\includegraphics[width=0.4\textwidth]{telecomparis.png} \par}
        \vspace{1cm}
        {\bfseries\LARGE Télécom Paris \par}
        \vspace{1cm}
        \vspace{2cm}
        {\scshape\Huge Project Robust Key-Value Store \par}
        \vspace{1cm}
            {\itshape\Large Fundamentals of algorithms \par}
        \vfill
        \vspace{1cm}
            {\itshape\Large Professor: KUZNETSOV Petr \par}
        \vspace{2cm}

        {\Large Members: \par}
        {\Large KABIL Aymen \par}
        {\Large FAMÀ Daniele \par}
        {\Large BARAU  Elena\par}
        {\Large MARCULESCU Tudor\par}

        \vfill

        \vfill
        {\Large January 2026 \par}
    \end{center}
\end{titlepage}

\clearpage


\tableofcontents
\clearpage
\listoftables
\listoffigures


\clearpage

\newpage
\section{Introduction}

In this project we are proposing an implementation for the robust key-value store problem in a distributed
system. Our design is based on Akka, which is a middleware that allows to easily create actor based
systems in Java. The ideea is to be able to simulate multiple nodes of the system on a single machine
using multiple actors in Akka.

Our system is fault tolerant to actors crashing. The system is able to continue if the number of actors
that crashed is less than half of their total number. If the number of crashes exceeds this limit,
the system is not able to guarantee the correctness, due to the posibility of having partitions.

%%%%%%%%%%%%%%%%%%%%%%%%%%%%%%%%%%%%%%%%%%%%%%%%%%%%%%%%%%%%%%%%%%%

% 2. System description

%%%%%%%%%%%%%%%%%%%%%%%%%%%%%%%%%%%%%%%%%%%%%%%%%%%%%%%%%%%%%%%%%%
\newpage
\section{System description}

The key-value store distributed system is highly concurent, meaning that multiple clients can send
write or read requests at the same time. The non-trivial of concurrency happens when multiple clients
are sending write requests for the same key at the same time, therefore having the same timestamp. This
is explained in the following subsections. In order to understand what the read and write operations
do, the sequential specification of them is explained first.

\subsection{Sequential specification}

The state of a key-value store is a set of key-value pairs The state of a key-value store is a set of key-value pairs of the form (k, v), where k is an integer and v
is a value in a given value set (assume that values are also integers). The initial state is an empty set.
The system exports two operations:

- put(k, v) sets the value with key k to v (overwriting the old value if it is already in the set).

- get(k) returns the value of key k (the default value $\bot$ is returned is the key is not at the system).

\subsection{Actor behaviour description}

To write a new value v, a writer increments its local sequence number r, sends a read request [?, r]
to all the processes, waits for the responses of the type [v', t', r] from a majority (> n/2) of the
processes (we also say a quorum). Then it picks up the highest timestamp tmax in the set of received
responses, computes its new timestamp as t = tmax + 1 and sends a write request [v, t] to all the pro-
cesses in the system. The operation completes as soon as responses of the type [ack, v, t] are received
from a quorum.

To read the register value, a reader increments its sequence number r and sends a read
request [?, r] to every all the processes. As soon as responses of the type [v', t', r] are received
from a quorum, the operation selects the value vm equipped with the highest timestamp tm. If there are
multiple different values written with timestamp tm, the largest such value is chosen. To make sure
that subsequent read operations will not “miss” the returned value, the reader then sends a write request
[vm, tm], waits until a quorum confirms it by sending [ack, vm, tm], and only then returns v.

Notice that, as multiple values can be written with the same timestamp, the value is attached to the
acknowledgement sent to a write request. This is needed to ensure that the acknowledgement is indeed
sent to this write request and not to an earlier write request with the same timestamp but a different value
(this could happen if the operation is concurrent with multiple writes using the same timestamp).
An alternative solution is to maintain an additional sequence number attached to every write request a process issues.

%%%%%%%%%%%%%%%%%%%%%%%%%%%%%%%%%%%%%%%%%%%%%%%%%%%%%%%%%%%%%%%%%%%

% 3. Proof of correcteness

%%%%%%%%%%%%%%%%%%%%%%%%%%%%%%%%%%%%%%%%%%%%%%%%%%%%%%%%%%%%%%%%%%
\newpage
\section{Proof of correctness}

In order to prove the correcteness of our implementation, we need to show that it satisfies both
safety and liveness properties. To do so, we will first start by proving that the implementation
of our Process.java class is safe.

\subsection{Safety}

The system implemented is similar to an ABD algorithm, but in our case, there are multiple writers
and readers and the implemented registers are atomic. For simplicity, out of the k keys that would
be possible to be used in order to store values, we will focus on the case of a single key. Therefore,
both the writers and readers will be using k = 0 to store and retrieve values.

Since we have multiple writers and readers, one would need to prove that the system still behaves
as an atomic register. This would mean that the last written value is the one that is retrieved by
the next read operation. A depiction of such a scenario is shown in the figure

// Add figure showing 3 concurrent processes writing and reading the same key

// Proof about the the imposibility of reading an old value after a new one has been written


\subsection{Liveness}

// Proof that if less than half of the processes crash, the system is still able to make progress


\subsection{Conclusion on correctness}

If both safety and liveness are satisfied, then the system is argued to be correct.

%%%%%%%%%%%%%%%%%%%%%%%%%%%%%%%%%%%%%%%%%%%%%%%%%%%%%%%%%%%%%%%%%%

% 4. Performance evaluation

%%%%%%%%%%%%%%%%%%%%%%%%%%%%%%%%%%%%%%%%%%%%%%%%%%%%%%%%%%%%%%%%%%
\newpage
\section {Performance evaluation}

Performance evaluation of our implementation is based on measuring the time taken for all the read
and write operations to complete. There are multiple factors that can influence this time, such as
the total number of processes in the system, the number of write and read operations.

Our test suite is implemented such that N the number of processes in the system, f the number of
failed processes and M the number of read and write operations can be configured before starting the
program. N is varied between 3, 10 and 100, while f is set to be equal to 1, 4 or 49. M is also varied
between 3, 10 and 100.

Table \ref{tab:performance-obtained} shows the results of our performance evaluation:
\begin{table}[H]
\centering
\begin{tabular}{|c|c|c|c|}
\hline
\textbf{Number of processes N} & \textbf{Number of failed processes f} & \textbf{Number of operations } & \textbf{Latency [ms]} \\
\hline

3    & 1                 & 3           & 10        \\
3    & 1                 & 10          & 23        \\
3    & 1                 & 100         & 113       \\
10   & 4                 & 3           & 61        \\
10   & 4                 & 10          & 79        \\
10   & 4                 & 100         & 201       \\
100  & 49                & 3           & 231       \\
100  & 49                & 10          & 353       \\
100  & 49                & 100         & 1343      \\

\hline
\end{tabular}
\caption{Performance obtained}\label{tab:performance-obtained}
\end{table}


%%%%%%%%%%%%%%%%%%%%%%%%%%%%%%%%%%%%%%%%%%%%%%%%%%%%%%%%%%%%%%%%%%

% 5. Conclusion

%%%%%%%%%%%%%%%%%%%%%%%%%%%%%%%%%%%%%%%%%%%%%%%%%%%%%%%%%%%%%%%%%%
\newpage
\section{Conclusion}

To conclude our report, we can say that our implementation of a robust key-value store in a distributed
system is able to satisfy both safety and liveness properties, making it a reliable solution for
storing and retrieving data in a distributed environment. The use of Akka as a middleware allowed us
to easily create an actor-based system that can handle multiple clients concurrently, while also
being fault-tolerant to actor crashes. Overall, our project demonstrates the feasibility of building
a robust key-value store in a distributed system.


%%%%%%%%%%%%%%%%%%%%%%%%%%%%%%%%%%%%%%%%%%%%%%%%%%%%%%%%%%%%%%%%%%

% SECTION BIBLIOGRAPHY

%%%%%%%%%%%%%%%%%%%%%%%%%%%%%%%%%%%%%%%%%%%%%%%%%%%%%%%%%%%%%%%%

\newpage % keep bibliography on a new page
\bibliographystyle{plain}   % choose a style
\bibliography{ref}      % name of .bib file (no .bib extension)


\end{document}